\documentclass{beamer}
%\usetheme{Madrid} % My favorite!
%\usetheme{Boadilla} % Pretty neat, soft color.
%\usetheme{default}
%\usetheme{Warsaw}
%\usetheme{Bergen} % This template has nagivation on the left
\usetheme{Frankfurt} % Similar to the default 
%with an extra region at the top.
%\usecolortheme{seahorse} % Simple and clean template
%\usetheme{Darmstadt} % not so good
% Uncomment the following line if you want %
% page numbers and using Warsaw theme%
 \setbeamertemplate{footline}[page number]
%\setbeamercovered{transparent}
\setbeamercovered{invisible}
% To remove the navigation symbols from 
% the bottom of slides%
\setbeamertemplate{navigation symbols}{ }
%
  \usepackage{graphicx}
  \usepackage{lmodern} %this font kills an annoying compliation warning, and is almost the same as the default.
%\usepackage{bm}         % For typesetting bold math (not \mathbold)
%\logo{\includegraphics[height=0.6cm]{yourlogo.eps}}
%
  \title{Measuring Conspicuous Consumption: a cross-country comparison}
  \author{David Jinkins}
\institute
{
Pennsylvania State University\\
\medskip
{\emph{david.jinkins@gmail.com}}
}
\date{\today}
% \today will show current date. 
% Alternatively, you can specify a date.
%
\begin{document}
%
\begin{frame}
\titlepage
\end{frame}
%
\section{Introduction}

\begin{frame}
  \frametitle{Motivation}
  	\begin{itemize}
           \item Signaling with consumption
	     \begin{itemize}
	       \item Why do people buy expensive watches? 
	       \item This paper: to signal well-being.
	     \end{itemize}
	   \item Well-being (wealth) is unobservable
	     \begin{itemize}
		\item In this paper, your social circle judges your well-being based on your consumption of a single good.
		\item Example: iPad in one social circle, brand-name clothing in another, vacations to exotic locales in another.
	     \end{itemize}
	   \item Why do people care about social beliefs?
	     \begin{itemize}
	       \item Maybe they just do--ex: post-mortem donations.
	       \item Maybe social beliefs are a stepping stone.
	       \item If first, this is a structural model.  If second, this is a reduced form model.
	     \end{itemize}
	\end{itemize}
\end{frame}
%
\begin{frame}
  \frametitle{Preview of Results}
  	\begin{itemize}
           \item Estimation of utility parameters  
	     \begin{itemize}
	       \item The utility function will loosely look like this:
		 \[ 
		   (1-\alpha) u(C) + \alpha E(u|C)
		 \]
	         The first term is fundamental utility, and the second is social belief.
	       \item Americans care about utils of social belief about 1/6 as much as they care about utils of consumption ($\alpha = .1458$).
	       \item Chinese care about utils of social belief about 1/4 as much as they care about utils of consumption ($\alpha = .2$).
	     \end{itemize}
	   \item Taxes on visible goods
	     \begin{itemize}
		\item I propose a tax on visible goods which dramatically increases social welfare.
		\item Median welfare increase is XX\%.
		\item Almost pareto efficient - only XX\% harmed.
	     \end{itemize}
	\end{itemize}
\end{frame}
%
\begin{frame}
  \frametitle{Recent Related literature}
  	\begin{itemize}
           \item Theory of consumption signaling:  
	     \begin{itemize}
	       \item Ireland(1994,JPubEcon),	     
	       \item Heffetz(2007,mimeo)
	     \end{itemize}
	   \item Empirical studies of consumption signaling: 
	     \begin{itemize}
		\item Charles, Hurst, and Roussanov(2009,QJE),
		\item  Heffetz(2012,REStat) 
	     \end{itemize}
	   \item Relative consumption and social status
	     \begin{itemize}
		\item Luttmer(2004,mimeo),
		\item  Arrow and Dasgupta(2009,TheEconJrnl), 
		\item  Clark, Frijters, and Shield(2008,JEL)
	     \end{itemize}
	\end{itemize}
\end{frame}
%
\section{Model}
\begin{frame}
  \frametitle{Environment}
  \begin{itemize} 
    \item Wealth is exogenous.
    \item There are I goods, and no saving.
    \item The price vector P is exogenous.
    \item Consumers choose a consumption vector C.
    \item Preferences differ across consumers, but are known within the social circle.
    \item Within each social circle, only expenditures on a single good category are observable. 
  \end{itemize}
\end{frame}
% 
\begin{frame}
  \frametitle{Preferences}
   \begin{itemize} 
     \item Social beliefs are described by the I functions $g_i:c_i,\Theta \rightarrow C$
    \item Following Ireland(1994) and Heffetz(2007), utilty has the following form:
      \[
	U(C,\theta,i) = (1-\alpha) u(C,\theta) + \alpha u(g_i(c_i,\theta),\theta)
      \]
    \item $u$ is called the fundamental utility function.
    \item $\theta$ is the preference heterogeneity.
  \end{itemize}
\end{frame}
%
\begin{frame}
  \frametitle{Equilibrium concept}
   \begin{itemize} 
     \item An equilibrium is a set of I belief functions $\{g_i\}$ and a set of $I$ consumption functions $\{C^i\}$ such that:  
       \begin{enumerate}
	 \item For all $i,\theta$, $C^i(\theta)$ solves the consumer's problem given $g_i$. 
	 \item For all $i,\theta$, $g_i(c_i^i(\theta),\theta) = C^i(\theta)$
       \end{enumerate}
     \item This is a standard ``separating equilbrium'' ala Spence.
  \end{itemize}
\end{frame}
%
\begin{frame}
  \frametitle{Solving the Model}
  \begin{itemize}
    \item Substituting optimal unobserved expenditures into the individual's problem, with some manipulation we can write:
  \end{itemize}
      \[
	U(C,\theta) = \theta_v \ln C_v + \left(1-\alpha \right)\hat{\theta} \ln (W-P_v C_v)  + \alpha \hat{\theta} \ln\left((h_v(C_v,\theta)\right) + \psi
      \]
      \hspace{.75cm} $\hat{\theta}$ and $\psi$ are known functions of $\theta$. $h_v$ is belief about $W-P_v C_v$.
      \begin{itemize}
    \item The FOC is then:
      \[
	h_v'(C_v,\theta) = \frac{1}{\alpha}\left(\left( 1-\alpha \right)P_v-\frac{\theta_v}{\hat{\theta}}\frac{h(C_v)}{C_V}  \right)	
      \]
    \item The solution to this differential equation is:
      \[
	h(C_v) = \frac{\hat{\theta}(1-\alpha)}{\theta_v + \alpha \hat{\theta}}P_v C_v + K C_v^{\frac{\theta_v}{\alpha \hat{\theta}}} 
      \]
    \item $K$ is pinned down by lowest possible wealth level.
  \end{itemize}
 \end{frame}
 %
 \section{Estimation}
\begin{frame}
  \frametitle{Assumptions}
  \begin{itemize}
    \item Primary goal is to get $\alpha$, the importance of signalingin utility.
    \item Assume that Cobb-Douglas parameters $\theta$ are distributed independent log-normal with a mass-point at zero.
    \item The mass-point is necessary because data is quite sparse.
    \item The log-normal assumption is due to shape of the data.
  \end{itemize}
 \end{frame}
%  
 \begin{frame}
   \centerline{The End}
 \end{frame}
 % End of slides
 \end{document}
